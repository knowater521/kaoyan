% by ziyuang@gmail.com

\documentclass{article}
\usepackage{amsmath,amssymb}
\usepackage{xltxtra,xunicode}
\usepackage{fontspec}

\setmainfont[BoldFont=SimHei]{SimSun}
\XeTeXlinebreaklocale "zh"
\XeTeXlinebreakskip = 0pt plus 1pt

\begin{document}
\begin{enumerate}
\item
\begin{enumerate}
\item
%\begin{problem}
设$a=(a_1,a_2,\cdots,a_n)$,$b=(b_1,b_2,\cdots,b_n)\in\mathbb{R}^n$,计算矩阵$A=I-a^\mathrm{T}b$的行列式.
%\end{problem}
\item
%\begin{problem}
设$n$阶实矩阵$A$的主对角元都为$0$,其余元素都为$1$. 求$A$的特征值和特征向量.
%\end{problem}
\item
%\begin{problem}
设$\displaystyle A=\begin{pmatrix}
2 & -1 & 0 & 1\\
0 & 3 & -1 & 0\\
0 & 1 & 1 & 0\\
0 & -1 & 0 & 3
\end{pmatrix}$. 求$A$的若当标准型.
%\end{problem}
\item
%\begin{problem}
设$f(x)=x^6-7x^4+8x^3-7x+7$,$g(x)=3x^5-7x^3+3x^2-7$. 求$f$与$g$的首一最大公因式$(f,g)$.
%\end{problem}
\item
%\begin{problem}
设$\displaystyle A=\begin{pmatrix}
3 & 1 & 1\\
2 & 4 & 2\\
-1 & -1 & 1
\end{pmatrix}$. 计算$A^{10}$
%\end{problem}
\item
%\begin{problem}
设$\displaystyle A=\begin{pmatrix}
3 & 0 & 0\\
0 & 2 & 1\\
0 & 1 & 2
\end{pmatrix}$. 求一正交矩阵$Q$使得$Q^{-1}AQ$为对角矩阵,并求正定矩阵$B$使得$A=B^2$.
%\end{problem}
\end{enumerate}
\item
\begin{enumerate}
\item
%\begin{problem}
令$S=\{(t,t^2,t^3):t\in\mathbb{R}\}$. 证明:$\mathrm{span}(S)=\mathbb{R}^3$.
%\end{problem}
\item
%\begin{problem}
设$A,B\in M_n(\mathbb{R})$,$A^2=A$,$B^2=B$,且$I-(A+B)$可逆. 证明:$A$与$B$的秩相等.
%\end{problem}
\item
%\begin{problem}
令$S=\{AB-BA:A,B\in M_n(F)\}$. 证明:$S$张成的子空间$W=\mathrm{span}(S)$的维数等于$n^2-1$,并且给出它的一个基.
%\end{problem}
\item
%\begin{problem}
设$V$为数域$F$上的线性空间,$W$是$V$的子空间. $V^*$表示$V$的对偶空间,$W^0$表示$W$的零化子,即$W^0=\{f\in V^*:f(W)=0\}$. 证明:$W^*\cong V^*/W^0$.
%\end{problem}
\item
%\begin{problem}
设$q(X)=X^\mathrm{T}AX$为$n$元实二次型. 如果$A$的所有特征值都属于区间$[a,b]$. 证明:$A-tI$对应的二次型当$t>b$时是负定的;当$t<a$时是正定的.
%\end{problem}
\item
%\begin{problem}
设$\sigma$是$n$维欧氏空间$V$的一个正规变换,且满足条件:$\sigma^2=id_V$,其中$id_V$为$V$上的恒等变换. 证明:$\sigma$既是对称变换,也是正交变换.
%\end{problem}
\end{enumerate}
\end{enumerate}
\end{document}