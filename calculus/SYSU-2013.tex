% by ziyuang@gmail.com

\documentclass{article}
\usepackage{amsmath,amssymb}
\usepackage{xltxtra,xunicode}
\usepackage{fontspec}

\setmainfont[BoldFont=SimHei]{SimSun}

\XeTeXlinebreaklocale "zh"
\XeTeXlinebreakskip = 0pt plus 1pt

\begin{document}
\begin{enumerate}
\item
\begin{enumerate}
\item
%\begin{problem}
设$\displaystyle x_n=\sqrt[\uproot{10}n]{\left[1+\left(\frac{1}{n}\right)^2\right]\left[1+\left(\frac{2}{n}\right)^2\right]\cdots\left[1+\bigg(\frac{n}{n}\bigg)^2\right]}$,求$\displaystyle\lim_{n\to\infty}x_n$.
%\end{problem}
\item
%\begin{problem}
求$\displaystyle\lim_{n\to\infty}n^2\left(x^{\frac{1}{n}}-x^{\frac{1}{n+1}}\right)$,其中$x>0$.
%\end{problem}
\item
%\begin{problem}
求 $\displaystyle \lim_{m\to\infty}\frac{\displaystyle\sum_{i=1}^{m}i^d-\frac{m^{d+1}}{d+1}}{\displaystyle m^d}$,其中$d>0$.
%\end{problem}
\end{enumerate}
\item
\begin{enumerate}
\item
%\begin{problem}
叙述数列$\{a_n\}$收敛的柯西收敛准则并证明之.
%\end{problem}
\item
%\begin{problem}
用柯西收敛准则证明:数列$\displaystyle a_n=\frac{1}{2\ln 2}+\frac{1}{3\ln 3}+\cdots+\frac{1}{n\ln n}$趋于无穷大.
%\end{problem}
\end{enumerate}
\item
\begin{enumerate}
\item
%\begin{problem}
证明$f(x)=\sin\sqrt{x}$在$[0,+\infty)$上一致连续.
%\end{problem}
\item
%\begin{problem}
证明$g(x)=\sin x^2$在$[0,+\infty)$上不一致连续.
%\end{problem}
\end{enumerate}
\item
%\begin{problem}
设$x_1=-1$,$\displaystyle x_{n+1}=-1+\frac{x_n^2}{2}\,(n=1,2,\cdots)$,证明$\displaystyle \lim_{n\to\infty}x_n$存在.
%\end{problem}
\item
%\begin{problem}
设$a_n>0\,(n=1,2,\cdots)$,证明$\displaystyle \varlimsup_{n\to\infty}n\left(\frac{1+a_{n+1}}{a_n}-1\right)\ge 1$.
%\end{problem}
\item
%\begin{problem}
设$0<x<1$,求$\displaystyle S(x)=\sum_{k=1}^{+\infty}x^k(1-x)^{2k}$的极值.
%\end{problem}
\item
%\begin{problem}
计算$\displaystyle \int_C \frac{(x+y)\,dx-(x-y)\,dy}{x^2+y^2}$,其中$C$是一条从$(-1,0)$到$(1,0)$不经过原点的光滑曲线:$y=f(x),-1\le x\le 1$.
%\end{problem}
\item
%\begin{problem}
计算$\displaystyle \int_S yx\,dxdy+zx\,dxdy+xy\,dxdy$,其中$S$是由$x^2+y^2=1$,三个坐标平面及$z=2-x^2-y^2$所围立体图形在第一卦限的外侧.
%\end{problem}
\item
%\begin{problem}
讨论级数$\displaystyle \sum_{k=1}^{+\infty}\frac{\sin kx}{k}$在$[0,2\pi]$上的一致连续性.
%\end{problem}
\item
\begin{enumerate}
\item
%\begin{problem}
分别将函数$\displaystyle f(x)=\frac{\pi - x}{2}$和$\displaystyle g(x)=
\begin{cases}
(\pi-1)x & 0\le x\le 1\\
\pi-x & 1 < x \le \pi
\end{cases}$
在$[0,\pi]$按正弦(Fourier)级数展开.
%\end{problem}
\item
%\begin{problem}
证明$\displaystyle \sum_{n=1}^{+\infty}\frac{\sin n}{n}=\sum_{n=1}^{+\infty}\left(\frac{\sin n}{n}\right)^2$.
%\end{problem}
\end{enumerate}
\end{enumerate}
\end{document}