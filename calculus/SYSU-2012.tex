% by ziyuang@gmail.com

\documentclass{article}
\usepackage{amsmath,amssymb}
\usepackage{xltxtra,xunicode}
\usepackage{fontspec}

\setmainfont[BoldFont=SimHei]{SimSun}

\begin{document}
\begin{enumerate}
\item
\begin{enumerate}
\item
求$\displaystyle\lim_{x\to 0}\left(1+x^2e^x\right)^{\frac{1}{1-\cos x}}$.
\item
给定$a_0$,$a_1$,并设$\displaystyle a_n=\frac{1}{2}\left(a_{n-1}+a_{n-2}\right)$,$n\ge 2$,求:$\displaystyle \lim_{n\to+\infty}a_n$.
\item
求 $\displaystyle I_n=\int_{0}^{n\pi}x|\sin x|\,dx$
\item 设 $g(x)$,$f(x,y)$ 均二阶可微,$u(x,y)=yg(\cos x)+f(e^x,xy)$,求$\displaystyle \frac{\partial u}{\partial x}$,$\displaystyle\frac{\partial^2 u}{\partial x\partial y}$.
\item
已知二椭圆抛物面为 $\Sigma_1:z=x^2+2y^2+1$,$\Sigma_2:z=2(x^2+3y^2)$,计算 $\Sigma_1$ 被 $\Sigma_2$  截下部分的曲面面积.
\item
求曲线积分 $\displaystyle \oint_{C}\frac{(x+4y)\,dy+(x-y)\,dx}{x^2+4y^2}$,其中$C$为以原点为圆心的单位圆,并取正向.
\item
判断级数 $\displaystyle \sum_{n=1}^{+\infty}n!\left(-\frac{e}{n}\right)^n$的敛散性.
\item
设$a_n>0$,$\displaystyle \lim_{n\to+\infty}a_n=a>0$,讨论级数$\displaystyle \sum_{n=1}^{+\infty}\left(\frac{a}{a_n}\right)^n$的敛散性.
\end{enumerate}
\item
给出函数$f(x)=x[x^{-1}]$在$(0,+\infty)$上的不连续点,其中$[x^{-1}]$表示$x^{-1}$的整数部分.
\item
设
\[
f(x,y)=
\begin{cases}
\displaystyle\frac{x^\alpha y}{x^2+y^2},&(x,y)\ne(0,0)\\
0,&(x,y)=(0,0)
\end{cases},
\]
且$\alpha>0$.问$\alpha$取何值时能使$f(x,y)$在点$(0,0)$可微?
\item
计算曲面积分
\[
\oint_L(y+1)\,dx+(z+2)\,dy+(x+3)\,dz
\]
其中$L$是球面 $x^2+y^2+z^2=R^2$ 被平面 $x+y+z=0$ 所截得的圆周.从$x$轴正向看去,$L$是逆时针方向.
\item
讨论函数项级数 $\displaystyle \sum_{n=1}^{+\infty}\frac{x^n}{n\ln n}$ 在$[0,1)$上的一致收敛性.
\item 设$f(x)$在$[a,b]$上有二阶连续导数,$\displaystyle f\left(\frac{a+b}{2}\right)=0$,求证:
\[
\left|\int_{a}^{b}f(x)\,dx\right|\le\frac{(b-a)^3}{24}\max_{a\le x\le b}\left|f''(x)\right|.
\]
\item 证明不等式 $yx^y(1-x)<e^{-1}$,其中$(x,y)\in D=\{(x,y)\mid 0<x<1,y>0\}$.
\item 设$x>a$时$g(x)>0$,$f(x)$和$g(x)$在任意有限区间$[a,b]$上可积,$\displaystyle \int_{a}^{+\infty}g(x)\,dx$发散,且$\displaystyle \lim_{x\to+\infty}\frac{f(x)}{g(x)}=0$,证明:$\displaystyle \lim_{x\to+\infty}\frac{\displaystyle\int_a^xf(t)\,dt}{\displaystyle\int_a^xg(t)\,dt}=0$.
\end{enumerate}
\end{document}